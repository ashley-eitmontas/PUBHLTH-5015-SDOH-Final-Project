% Options for packages loaded elsewhere
\PassOptionsToPackage{unicode}{hyperref}
\PassOptionsToPackage{hyphens}{url}
\documentclass[
]{article}
\usepackage{xcolor}
\usepackage[margin=1in]{geometry}
\usepackage{amsmath,amssymb}
\setcounter{secnumdepth}{-\maxdimen} % remove section numbering
\usepackage{iftex}
\ifPDFTeX
  \usepackage[T1]{fontenc}
  \usepackage[utf8]{inputenc}
  \usepackage{textcomp} % provide euro and other symbols
\else % if luatex or xetex
  \usepackage{unicode-math} % this also loads fontspec
  \defaultfontfeatures{Scale=MatchLowercase}
  \defaultfontfeatures[\rmfamily]{Ligatures=TeX,Scale=1}
\fi
\usepackage{lmodern}
\ifPDFTeX\else
  % xetex/luatex font selection
\fi
% Use upquote if available, for straight quotes in verbatim environments
\IfFileExists{upquote.sty}{\usepackage{upquote}}{}
\IfFileExists{microtype.sty}{% use microtype if available
  \usepackage[]{microtype}
  \UseMicrotypeSet[protrusion]{basicmath} % disable protrusion for tt fonts
}{}
\makeatletter
\@ifundefined{KOMAClassName}{% if non-KOMA class
  \IfFileExists{parskip.sty}{%
    \usepackage{parskip}
  }{% else
    \setlength{\parindent}{0pt}
    \setlength{\parskip}{6pt plus 2pt minus 1pt}}
}{% if KOMA class
  \KOMAoptions{parskip=half}}
\makeatother
\usepackage{color}
\usepackage{fancyvrb}
\newcommand{\VerbBar}{|}
\newcommand{\VERB}{\Verb[commandchars=\\\{\}]}
\DefineVerbatimEnvironment{Highlighting}{Verbatim}{commandchars=\\\{\}}
% Add ',fontsize=\small' for more characters per line
\usepackage{framed}
\definecolor{shadecolor}{RGB}{248,248,248}
\newenvironment{Shaded}{\begin{snugshade}}{\end{snugshade}}
\newcommand{\AlertTok}[1]{\textcolor[rgb]{0.94,0.16,0.16}{#1}}
\newcommand{\AnnotationTok}[1]{\textcolor[rgb]{0.56,0.35,0.01}{\textbf{\textit{#1}}}}
\newcommand{\AttributeTok}[1]{\textcolor[rgb]{0.13,0.29,0.53}{#1}}
\newcommand{\BaseNTok}[1]{\textcolor[rgb]{0.00,0.00,0.81}{#1}}
\newcommand{\BuiltInTok}[1]{#1}
\newcommand{\CharTok}[1]{\textcolor[rgb]{0.31,0.60,0.02}{#1}}
\newcommand{\CommentTok}[1]{\textcolor[rgb]{0.56,0.35,0.01}{\textit{#1}}}
\newcommand{\CommentVarTok}[1]{\textcolor[rgb]{0.56,0.35,0.01}{\textbf{\textit{#1}}}}
\newcommand{\ConstantTok}[1]{\textcolor[rgb]{0.56,0.35,0.01}{#1}}
\newcommand{\ControlFlowTok}[1]{\textcolor[rgb]{0.13,0.29,0.53}{\textbf{#1}}}
\newcommand{\DataTypeTok}[1]{\textcolor[rgb]{0.13,0.29,0.53}{#1}}
\newcommand{\DecValTok}[1]{\textcolor[rgb]{0.00,0.00,0.81}{#1}}
\newcommand{\DocumentationTok}[1]{\textcolor[rgb]{0.56,0.35,0.01}{\textbf{\textit{#1}}}}
\newcommand{\ErrorTok}[1]{\textcolor[rgb]{0.64,0.00,0.00}{\textbf{#1}}}
\newcommand{\ExtensionTok}[1]{#1}
\newcommand{\FloatTok}[1]{\textcolor[rgb]{0.00,0.00,0.81}{#1}}
\newcommand{\FunctionTok}[1]{\textcolor[rgb]{0.13,0.29,0.53}{\textbf{#1}}}
\newcommand{\ImportTok}[1]{#1}
\newcommand{\InformationTok}[1]{\textcolor[rgb]{0.56,0.35,0.01}{\textbf{\textit{#1}}}}
\newcommand{\KeywordTok}[1]{\textcolor[rgb]{0.13,0.29,0.53}{\textbf{#1}}}
\newcommand{\NormalTok}[1]{#1}
\newcommand{\OperatorTok}[1]{\textcolor[rgb]{0.81,0.36,0.00}{\textbf{#1}}}
\newcommand{\OtherTok}[1]{\textcolor[rgb]{0.56,0.35,0.01}{#1}}
\newcommand{\PreprocessorTok}[1]{\textcolor[rgb]{0.56,0.35,0.01}{\textit{#1}}}
\newcommand{\RegionMarkerTok}[1]{#1}
\newcommand{\SpecialCharTok}[1]{\textcolor[rgb]{0.81,0.36,0.00}{\textbf{#1}}}
\newcommand{\SpecialStringTok}[1]{\textcolor[rgb]{0.31,0.60,0.02}{#1}}
\newcommand{\StringTok}[1]{\textcolor[rgb]{0.31,0.60,0.02}{#1}}
\newcommand{\VariableTok}[1]{\textcolor[rgb]{0.00,0.00,0.00}{#1}}
\newcommand{\VerbatimStringTok}[1]{\textcolor[rgb]{0.31,0.60,0.02}{#1}}
\newcommand{\WarningTok}[1]{\textcolor[rgb]{0.56,0.35,0.01}{\textbf{\textit{#1}}}}
\usepackage{longtable,booktabs,array}
\usepackage{calc} % for calculating minipage widths
% Correct order of tables after \paragraph or \subparagraph
\usepackage{etoolbox}
\makeatletter
\patchcmd\longtable{\par}{\if@noskipsec\mbox{}\fi\par}{}{}
\makeatother
% Allow footnotes in longtable head/foot
\IfFileExists{footnotehyper.sty}{\usepackage{footnotehyper}}{\usepackage{footnote}}
\makesavenoteenv{longtable}
\usepackage{graphicx}
\makeatletter
\newsavebox\pandoc@box
\newcommand*\pandocbounded[1]{% scales image to fit in text height/width
  \sbox\pandoc@box{#1}%
  \Gscale@div\@tempa{\textheight}{\dimexpr\ht\pandoc@box+\dp\pandoc@box\relax}%
  \Gscale@div\@tempb{\linewidth}{\wd\pandoc@box}%
  \ifdim\@tempb\p@<\@tempa\p@\let\@tempa\@tempb\fi% select the smaller of both
  \ifdim\@tempa\p@<\p@\scalebox{\@tempa}{\usebox\pandoc@box}%
  \else\usebox{\pandoc@box}%
  \fi%
}
% Set default figure placement to htbp
\def\fps@figure{htbp}
\makeatother
\setlength{\emergencystretch}{3em} % prevent overfull lines
\providecommand{\tightlist}{%
  \setlength{\itemsep}{0pt}\setlength{\parskip}{0pt}}
\usepackage{bookmark}
\IfFileExists{xurl.sty}{\usepackage{xurl}}{} % add URL line breaks if available
\urlstyle{same}
\hypersetup{
  pdftitle={Final Project Report},
  pdfauthor={Ashley Eitmontas},
  hidelinks,
  pdfcreator={LaTeX via pandoc}}

\title{Final Project Report}
\author{Ashley Eitmontas}
\date{2025-12-07}

\begin{document}
\maketitle

\subsection{Introduction}\label{introduction}

Asthma remains a pressing public health concern across the United
States, particularly within rural communities which often experience
disproportionately high rates of the chronic disease. Asthma, a chronic
inflammatory disease of the lungs, is shaped not only by biology but
also by structural inequalities. In Ohio, the lifetime asthma prevalence
rates are significant, affected 16.09\% of adults and 9.73\% of
children, while current asthma affects 11.39\% of adults and 6.81\% of
children (ODH).

In Ohio, particularly in rural and Appalachian counties, factors such as
economic hardship and limited healthcare access contribute to elevated
asthma prevalence. This study's purpose is to identify which social
determinants of health (SDOH) most strongly predict asthma prevalence
across these counties using publicly available data indicators from
DataOhio.

\subsection{Data Description}\label{data-description}

This study utilized the Social Determinants of Health data set sourced
from DataOhio, which provided county-level SDOH indicators. The analysis
was limited to 32 Appalachian counties in Ohio, including the following:

\emph{Adams County, Ashtabula County, Athens County, Belmont County,
Brown County, Carroll County, Clermont County, Columbiana County,
Coshocton County, Gallia County, Guernsey County, Harrison County,
Highland County, Hocking County, Holmes County, Jackson County,
Jefferson County, Lawrence County, Mahoning County, Meigs County, Monroe
County, Morgan County, Muskingum County, Noble County, Perry County,
Pike County, Ross County, Scioto County, Trumbull County, Tuscarawas
County, Vinton County}

The selected predictor variables examined in this study include poverty
rate, median household income, unemployment rate, educational attainment
(defined as the rate of individuals 25 years or older with less than a
high school degree), and uninsured rate (which measures healthcare
access). The outcome variable studies was the asthma prevalence rate.

\subsection{Methods}\label{methods}

The primary analysis approach utilized in this study was Multiple Linear
Regression (MLR) implemented in R, focusing on county-level data. We had
two central aims to the study. Firstly, aim one was to use MLR to
measure the relationship between asthma rates and the key predictors
(poverty, insurance coverage, unemployment, income, and education). The
second aim was to determine which social determinants are statistically
significant drivers of asthma disparities in Appalachian counties in
Ohio. Model selection was guided by theory and diagnostics, including
multicollinearity diagnostics. Specifically, Backward Elimination
Multiple Linear Regression was employed to narrow the final set of
predictors. The strength and reliability of the associations were
assessed using the adjusted R\textsuperscript{2}, p-values, and residual
diagnostics.

\subsection{Results}\label{results}

\paragraph{Multiple Linear Regression Backward Elimination
Findings}\label{multiple-linear-regression-backward-elimination-findings}

The full multiple linear regression model relating the selected SDOH
indicators to asthma prevalence was satistically significant, with a
p-value of 0.0001121. The adjusted R\textsuperscript{2} value was
.05305, which indicates that the predictors retained (poverty rate and
median household income) explain approximately 53\% if the variation in
asthma prevalence across the Appalachian counties.

\emph{*Insert tables and figures generated by code here}*

\begin{verbatim}
## Warning: package 'readr' was built under R version 4.5.2
\end{verbatim}

\begin{verbatim}
## Rows: 0 Columns: 1
## -- Column specification --------------------------------------------------------
## Delimiter: ","
## chr (1): PUBHLTH-5015-SDOH-Final-Project/results/model_df.csv
## 
## i Use `spec()` to retrieve the full column specification for this data.
## i Specify the column types or set `show_col_types = FALSE` to quiet this message.
\end{verbatim}

\begin{longtable}[]{@{}l@{}}
\caption{Sample of modeling dataset (first 10 rows)}\tabularnewline
\toprule\noalign{}
PUBHLTH-5015-SDOH-Final-Project/results/model\_df.csv \\
\midrule\noalign{}
\endfirsthead
\toprule\noalign{}
PUBHLTH-5015-SDOH-Final-Project/results/model\_df.csv \\
\midrule\noalign{}
\endhead
\bottomrule\noalign{}
\endlastfoot
\end{longtable}

\begin{Shaded}
\begin{Highlighting}[]
\NormalTok{knitr}\SpecialCharTok{::}\FunctionTok{include\_graphics}\NormalTok{(}\StringTok{"../figs/asthma\_choropleth.png"}\NormalTok{)}
\end{Highlighting}
\end{Shaded}

\includegraphics[width=33.33in]{../figs/asthma_choropleth}

\begin{Shaded}
\begin{Highlighting}[]
\NormalTok{knitr}\SpecialCharTok{::}\FunctionTok{include\_graphics}\NormalTok{(}\StringTok{"../figs/scatter\_asthma\_vs\_poverty\_rate.png"}\NormalTok{)}
\end{Highlighting}
\end{Shaded}

\includegraphics[width=29.17in]{../figs/scatter_asthma_vs_poverty_rate}

\begin{Shaded}
\begin{Highlighting}[]
\NormalTok{knitr}\SpecialCharTok{::}\FunctionTok{include\_graphics}\NormalTok{(}\StringTok{"../figs/scatter\_asthma\_vs\_median\_household\_income.png"}\NormalTok{)}
\end{Highlighting}
\end{Shaded}

\includegraphics[width=29.17in]{../figs/scatter_asthma_vs_median_household_income}

\begin{Shaded}
\begin{Highlighting}[]
\NormalTok{knitr}\SpecialCharTok{::}\FunctionTok{include\_graphics}\NormalTok{(}\StringTok{"../figs/model\_diagnostics.png"}\NormalTok{)}
\end{Highlighting}
\end{Shaded}

\includegraphics[width=16.67in]{../figs/model_diagnostics}

\begin{Shaded}
\begin{Highlighting}[]
\NormalTok{knitr}\SpecialCharTok{::}\FunctionTok{include\_graphics}\NormalTok{(}\StringTok{"../figs/backward\_elimination\_path.png"}\NormalTok{)}
\end{Highlighting}
\end{Shaded}

\includegraphics[width=37.5in]{../figs/backward_elimination_path}

\paragraph{Individual County Metrics}\label{individual-county-metrics}

When linear regressions were run for each predictor on a county level,
poverty and income showed the most consistent and significant results.
Poverty Rate was significant in 22 out of 32 counties, and Median
Household Income was significant in 20 out of 32 counties. This finding
suggests that these two metrics are the strongest predictors when
looking at individual regressions for each county. Additionally, the
larger counties tended to have more significant results for each metric
regression, likely due to a larger data set contributing to more
accurate results.

\emph{*Insert tables and figures generated by code here}*

\subsection{Conclusions}\label{conclusions}

Asthma outcomes in Appalachian Ohio counties are strongly shaped by
structural determinants: poverty and income. These two metrics were the
strongest predictors when looking at individual county regressions and
were the only significant predictors retained in the final backward
elimination multiple linear regression model. The findings indicate that
asthma is likely related to poorer living conditions and other adverse
health outcomes associated with poverty. These results produced rich
implications for future research and policy. The fact that
non-significant predictors included the uninsured rate and unemployment
rate suggests that the broader socioeconomic context outweighs
individual access measures in determining asthma prevalence at the
county level in Appalachian Ohio. Therefore, tackling asthma disparities
requires equity-focused strategies that move beyond healthcare access
alone. Policy and public health efforts should look more upstream and
focus on poverty reduction and economic development within Appalachian
communities to address the root causes of inequity.

As with any study, it is also essential to include limitations within
this research. The analysis is observational and cross-sectional,
meaning it can only establish associations, not causality. Furthermore,
the adjusted R\textsuperscript{2} of 0.5305 shows that approximately
47\% of the variation in asthma prevalence is unexplained by the
included variables, suggesting that other unmeasured factors--such as
environmental exposures (air quality, industrial sites, housing quality)
and temporal dynamics--likely play a major role in asthma prevalence.
The analysis is also limited to county-level data in Appalachian Ohio,
which could mask important neighborhood or household-level disparities,
and hinders generalization of the results to other non-Appalachian Ohio
counties.

\end{document}
